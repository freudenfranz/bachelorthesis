\documentclass[a4paper, 12pt, headsepline, numbers=noenddot, bibliography=totoc, index=totoc, fleqn, DIVcalc, headings=normal]{article}
% Standard-Dokument mit
%   Papierformat A4
%   Binderand 5 mm
%   Schrift 12-Punkt
%   Linie unter der Kopfzeile
%   Nummern ohne Punkt am Ende
%   Literaturverzeichnis mit Nummer im Inhaltsverzeichnis
%   Index mit Nummer im Inhaltsverzeichnis
%   Formeln werden linksb�ndig statt zentriert angeordnet
%   Berechne automatisch gute Werte f�r DIV
%	Etwas kleinere �berschriften

%Fonts etc:
%8-Bit-Fonts
\usepackage[T1]{fontenc}
%Schriftart
\usepackage{helvet}
% zus�tzliche Symbole
\usepackage{textcomp,latexsym}
%Latin1-Umlaute in der Eingabe
\usepackage[latin1]{inputenc}
%neu deutsche Rechtschreibung
\usepackage[ngerman]{babel}
%f�r die aktuelle Zeit
\usepackage{scrtime}
%f�r Definitionen, Theoreme u.�.
\usepackage{theorem}
\theoremstyle{break}
\newtheorem{Def}{Definition}
%Web-Adressen auch mit T1-Encoding
\usepackage[T1]{url}
%und in tt-Font
\urlstyle{tt}
%konfigurierbarer Zeilenabstand
\usepackage{setspace}
%Grafik
\usepackage{graphicx}
%Mathematik: wer's braucht, sollte folgende Zeilen verwenden:
\usepackage{amsfonts}
\usepackage{amsmath}
%Algorithm/Algorithmic-Umgebung f�r Algorithmen
\usepackage{algorithm}
\usepackage{algorithmic}
%Mehrere Bibliographien in einem Dokument
\usepackage{multibib}
\usepackage{hyperref}

%% Hyperref erzeugt Hyperlinks in PDF-Dokumenten und erlaubt das
%% Setzen bestimmter Attribute
%\usepackage[
%% Titel der Arbeit
%pdftitle={<Thema der Arbeit>},
%% Autor der Arbeit
%pdfauthor={<Name des Autoren>},
%% Art der Arbeit
%pdfsubject=<Typ der Arbeit>]{hyperref}

% fette Seitenzahlen im index f�r die Hauptreferenz erm�glichen
% Verwendung: \index{Stichwort|ibf}
\newcommand{\ibf}[1]{{\bfseries #1}}

% Wer's h�bscher findet: Numerierung von Formeln nicht mit
% Kapitelnummer.Formelnummer, sondern nur mit Formelnummer
%\renewcommand{\theequation}{\arabic{equation}}

%eine Randnote f�r fehlende Teile
\newcommand{\fehlt}[1]{(\marginpar[\hfill!$\longrightarrow$]{$\longleftarrow$!}Hier fehlt \emph{#1})}
%Keinen Einzug bei neuen Abs�tzen
 \setlength{\parindent}{0ex}
%Abstand zweier Listenelemente kleiner
\setlength{\itemsep}{0ex plus0.2ex}
%Benutze bei der Numerierung von Gleichungen "Kapitelnummer.Gleichungsnummer"
\renewcommand{\theequation}{\thechapter.\arabic{equation}}

% Datum
\date{Berlin\\ \today \\ \vspace{0.2cm}}

%Und hier geht es los =============================================================
\begin{document}

\vspace{4cm}
  \begin{center}

  \includegraphics[scale=0.2]{TuLogo}\\

  \vspace{0.2cm}

  {\scshape Fachgebiet Komplexe und Verteilte IT-Systeme}\\
  \vspace{1cm
  \includegraphics[scale=0.13]{logo}\\ 
}
%  {\scshape Institut f�r Softwaretechnik und Theoretische Informatik}
  \vspace{1cm}


  \bfseries{????????????????????????}
  \vspace{2cm}

  - EXPOS\'{E} -

  \vspace{4cm}
  \begin{tabular}{|p{5cm}|p{9cm}|}
  \hline
  Familienname,Vorname & Stimpfl, Jakob Lorenz\\ \hline
  Erstellung des Expos\'{e}s & \today\\ \hline
  Studiengang & Computer Engineering (B.Sc.)\\ \hline
  Betreuer & Prof. Dr. Kao, Jonas Winkler\\ \hline
  \end{tabular}

  \end{center}

\newpage

\section{Einleitung}
 

\section{Modes}
\cite{Vogelsang}. 

\begin{itemize}
\item 
\item 
\item
\end{itemize}


\section{Aufgabenstellung}
test\\
test\\ 
Folgende Teilaufgaben konnten daf�r identifiziert werden:

\begin{itemize}
\item 
\item 
\item 
\item 
\end{itemize}

\section{L�sungsansatz}

\section{Evaluation}

\bibliography{literature}
\bibliographystyle {plain}

\end{document}

